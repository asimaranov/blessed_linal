\documentclass[l1pt]{article}
\usepackage[top=0.5in, bottom=1.1in, right=0.8in]{geometry}
\usepackage[T2A]{fontenc}
\usepackage[utf8x]{inputenc}
\usepackage[english, russian]{babel}

\usepackage{amsmath}
\usepackage{sectsty}
\usepackage{amssymb}
\usepackage{lipsum}
\usepackage{amsfonts} 
\usepackage{enumitem}
\usepackage{ragged2e}
\usepackage{ntheorem}
\usepackage{mathtools}

\newtheorem{theorem}{Теорема}
\allsectionsfont{\centering}

\newenvironment{proof}{\paragraph{Доказательство:}}{\hfill\newline}


\begin{document}
\section*{Лекция 1}
\date{May 2017}
\subsection*{Тема: Линейное подпространтсво}
\begin{flushleft}
Пусть $\underline{V,  \mathbb{P}}$ - некоторое линейное пространство над полем $\mathbb{P}$

Тогда в нем выполняются законы композиции:
$\forall a, b \in V, \quad \forall \alpha \in \mathbb{P}$
\begin{enumerate}
 \item $a + b \in V \quad V \times V \rightarrow V$
 \item $\alpha * a \in V \quad \mathbb{P} \times V \rightarrow V$
\end{enumerate}


Линейные пространства $\underline{\mathbb{C}, \mathbb{C}}$ и $\underline{\mathbb{C}, \mathbb{R}}$ отличаются размерностью\newline

Пусть $b, a_1, \dots , a_n \in V$, $\alpha_1, \dots , \alpha_n \in \mathbb{P}$, где и b = $\alpha_1 a_1$ + $\alpha_2 a_2$ + \dots + $\alpha_2 a_2$
Тогда вектор b называется линейной комбинацией векторов $a_1, \dots , a_n$\newline

Линейная комбинация называется нетривиальной, если хотя бы один коэффициент не равен нулю

Линейная оболочка векторов $a_1, \dots , a_n \in V$ - множество всевозможных линейных комбинаций этих веторов. Обозначается как $L(a_1, \dots , a_n)$

Линейное пространство $W \neq \varnothing$ называется линейным подпространством пространства V, если
\begin{itemize}
 \item $W \subset V$
 \item оно само является линейным пространством относительно операций композиции из V 
\end{itemize}

Вектора $a_1, \dots , a_n \in V$ называют линейно зависимыми, если существует их нетривиальная линейная комбинация, равная нулевому вектору\newline

Линейное пространство называется бесконечномерным, если $\forall n \in \mathbb{N}$ найдется набор из n линейнонезависимых векторов. 
Примеры:
\begin{itemize}
 \item Множество функций, непрерывных на некотором промежутке
 \item Множество всех многочленов 
\end{itemize}

Рассмотрим линейное пространство многочленов $\underline{\mathbb{M_n}, \mathbb{R}}$ *Народ, а что Панф с ним сделал в итоге, используя факт непредставимости трансцендентных чисел в виде корня полинома? Я забыл*\newline

Пусть есть такое натуральное число m, что любые m + 1 векторов из V линейно зависимы. Очевидно, что любые m+1 векторов также линейно зависимы. Мы можем взять минимальное число из всех m (т.к ограниченное  снизу подмножество натуральных чисел имеет минимум). Назовем его n. Это число- dim V. Взяв n линейно независимых векторов, мы получаем базис V $e_1, \dots , e_n \in V$

Базисные вектора 
\begin{itemize}
 \item Линейно независимы
 \item Упорядоченные *Я забыл, для чего оно требуется, Панф про это отдельно говорил*
 \item $\forall a \in V \quad e_1, \dots , e_n, a$ линейно зависимы
\end{itemize}

$V = L(e_1, e_2, \dots, e_n)$ по свойствам базисных векторов (любой вектор выражается из базисных)



\begin{theorem}
    Пусть есть $\underline{V, \mathbb{P}}$ (конечномерное), W- подпространство. Тогда $dim W \leqslant dim V$
\end{theorem}
\begin{proof}
    Очевидно, ибо базисные вектора в W так же будут базисными векторами в V
\end{proof}

\begin{theorem}
    Пусть есть $\underline{V, \mathbb{P}}$ (конечномерное) , W- подпространство. $dim W = dim V \iff W = V$
\end{theorem}
\begin{proof}
    Базисные вектора V будут так же базисными векторами в V, и наоборот. Значит, они совпадают. Если определить V и W как линейные оболочки этих векторов, их равенство очевидно
\end{proof}

\begin{theorem}
    Пусть есть $\underline{V, \mathbb{P}}$ (конечномерное). Тогда набор векторов  $e_1, \dots , e_k$ либо базис, либо существует набор векторов $e_{k + 1}, \dots , e_n$ такой, что $e_1, \dots , e_k, e_{k+1}, \dots , e_n$ - базис
\end{theorem}
\begin{proof}
    Есть два случая 
    \begin{itemize}
        \item $L(e_1, \dots , e_k) = V$. Тогда $(e_1, \dots , e_k)$ - базис
        \item $\exists e_{k+1} \in V \setminus L(e_1, \dots , e_k)$ . В таком случае мы последовательно `добираем` вектора до базиса
\end{itemize}
\end{proof}\par

Пусть 
$
  \begin{array}{c@{}c@{}c}
    e_1, & \dots, & e_n \\
    f_1, & \dots, & f_n
  \end{array}
  \quad
   \text{- базисы в V}
$

Тогда некоторую координату x можно выразить как $x = ex_e$, или $x = fx_f$, где \newline

$
x_e = \begin{pmatrix}
    x_1 \\
    x_2 \\
    \dots \\
    x_n
    \end{pmatrix}
$,\quad
$x_f = \begin{pmatrix}
    y_1 \\
    y_2 \\
    \dots \\
    y_n
    \end{pmatrix}$
\newline\newline

Таким образом, 
$
\begin{cases}
   f_1 = C_{11}e_1 + C_{12}e_2 + \dots + C_{1n}e_n \\
   \dots \\
   f_n = C_{n1}e_1 + C_{n2}e_2 + \dots + C_{nn}e_n
 \end{cases}
$
\newline\newline

Т.е $f = e\,C$

$C = (C_{i, j})$ называют матрицей перехода. По построению, ее определитель не равен нулю, ибо ее столбцы состоят из базисных векторов, которые линейно независимы. Знак определителя задает ориентацию\newline \par

Пусть есть $\underline{V, \mathbb{P}}$, $W_1$, $W_2$- подпространства V. Введем следующие понятия

\begin{itemize}
 \item $W_1 \cup W_2$ - объединение подпространств
 \item $W_1 + W_2$ = $\{x_1 + x_2 \quad \forall x_1 \in W_1, \; \forall x_2 \in W_2\}$ - сумма подпространств
\end{itemize}

Утверждение: $W_1 \cup W_2$ и $W_1 + W_2$ - подпространства V

В качестве примера подпространство можно привести множество решений однородной системы уравнений Ax = 0 ($A \in \mathbb{P}^{m*n}$). Как доказывалось ранее, оно образует подпространство линейного пространства $\mathbb{P}^n$\newline

\begin{theorem}[Грассмана]
    Пусть есть $\underline{V, \mathbb{P}}$, $W_1$, $W_2$- подпространства V. \newlineТогда
    $\{\theta\}  \subset 
    W_1 \cup W_2 \subset 
    \begin{array}{c@{}c@{}c}
        W_1 \\
        W_2
    \end{array} \subset 
    W_1 + W_2 \subset 
    V
  $
\end{theorem}
\begin{proof}
    Завтра добавлю
\end{proof}
\end{flushleft}
\end{document}
